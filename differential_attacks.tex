

%%%%%%%%%%%%%%%%%%%%%%%%%%%%%%%%%%%%%%%%%%%%%%%%%%%%%%%%%%%%%%%%%%%
%                                                                 %
%                       Differential Attacks                      %
%                                                                 %
%%%%%%%%%%%%%%%%%%%%%%%%%%%%%%%%%%%%%%%%%%%%%%%%%%%%%%%%%%%%%%%%%%%


\chapter{Differential Attacks} \label{c:differential attacks}

So far we have taken a look at a number of sections, which for the most part
look fairly disjoint. In this section, we will tie these concepts together and
discuss the process whereby a differential attack can be mounted against a
block cipher. Perhaps the easiest way to do this would be by means of working
through the process in various subsections, and then putting it all together in
the form of a theoretical discussion of why it works.

So before we begin, we would want to clarify what assumptions need to be made
in order to mount a successful attack against the type of Block Cipher
discussed in the previous chapter.
\paragraph{}

We can reduce them to the following list:
\begin{enumerate}
\item We can choose some plaintext.
\item We know or can generate the corresponding ciphertext.
\item We know the S-boxes 
\item We do not know the key
\end{enumerate}

Accepting those assumptions, we are ready to note down our algorithm for
attacking Block Ciphers, and specifically a SPN, with Differential
Cryptanalysis.  So let's explain how we would attack an SPN then:

Firstly, we will choose some plaintext and generate the corresponding cipher
text by assumption 2. We now have an fairly large number of plaintext inputs
and their corresponding ciphertext outputs. What we are wanting to do however,
is search for any pairs of plaintext $(P, P')$ with XOR difference $\Delta P$,
which results in the corresponding ciphertext pair $(C, C')$ having XOR 
difference $\Delta C$ with high probability. In this way, we will be exploting
the fact that S-boxes do not affect these so called differentials. To be clear
however in the above, $E_K(P) = C$ and $E_K(P') = C'$. Futhermore, $P \oplus P'
= \Delta P$ and $C \oplus C' = \Delta C$ and the pair $(\Delta P, \Delta C)$ is
known as a differential.

If we perform this method over all the inputs that we have, we can tabulate
the trails that occur with high probability. In a perfect cipher, the probablity
of any particular output difference occuring for any particular input difference
is $\frac{1}{2^n}$ \cite[p. 19]{Heys} where $n$ is the block size, so we are
looking for pairs with probabilities much higher than this.

Thinking back to the structure of our SPN, if there is an S-box after the last
round, we can simply invert it and get what the input would be to  said S-box,
so either way we are left facing the last round key.  


In summary, our basic methodology is as follows:
\begin{enumerate}
\item Choose some plaintext/ciphertext pairs.
\item Compute the differentials and find differential trails with high probabilities 
\item Extract the last round key in one of two ways
\item Rinse and repeat
\end{enumerate}

Let us discuss each of these in greater details then.

\section{Choosing Plaintext/Ciphertext Pairs}

\section{Computing Differentials}
So how do we analyze an S-box to retrieve its differentials? Well since the S-boxes
are publicly known, we can iterate through all possible inputs $X$ to a particular S-box,
and for each input, iterate through all possible input differences $\Delta X$ to get 
input pairs $(X, X' = X \oplus \Delta X$. Then, by sending the inputs, $X$ and $X'$
through the S-box, we can calculate each possible output difference $\Delta Y$ given
by $\Delta Y = S(X) \oplus S(X')$. Once we have all of these output differences,
we can tally the output differences that occur often for a given input difference,
and record them in a differential table. 

This might be best explained using an example, so let us consider the S-box 
represented by the following table (in hexadecimal):
\begin{center}
\begin{tabular}{|r|r|}
\hline
$X$ & $S(X)$ \\\hline
0 & 3 \\\hline
1 & 5 \\\hline
2 & 2 \\\hline
3 & 1 \\\hline
4 & 6 \\\hline
5 & 7 \\\hline
6 & 4 \\\hline
7 & 0 \\\hline
8 & A \\\hline
9 & C \\\hline
A & B \\\hline
B & 8 \\\hline
C & 9 \\\hline
D & F \\\hline
E & D \\\hline
F & E \\\hline
\end{tabular}
\end{center}
You might notice that this S-box function values do not seem to be particularly
well distributed. In fact, any value less than 8 is mapped to corresponding
value less than 8, and consequently values greater than or equal to 8 are mapped
to values greater than 8. This is so that we get clear outliers in terms of
high-probabities differentials when we compute the differential table.

Thus, if we step through possible inputs, $\{0, ..., 15\}$ to our S-box, and for
each input, XOR it with all the possible input differences $\{0, ..., 15\}$, we
get the following pairs $(X, X \oplus \Delta X)$ represented by the row and
column headings of the table below:
\begin{comment}
\begin{center}
\begin{tabular}{|c|c|c|c|c|c|c|c|c|}
\hline
& 000 & 001 & 010 & 011 & 100 & 101 & 110 & 111 \\\hline
001 & & & & & & & & \\\hline
010 & & & & & & & & \\\hline
011 & & & & & & & & \\\hline
100 & & & & & & & & \\\hline
101 & & & & & & & & \\\hline
110 & & & & & & & & \\\hline
111 & & & & & & & & \\\hline
\end{tabular}
\end{center}
\end{comment}
\begin{center}
\begin{tabular}{c|c||c|c|c|c|c|c|c|c|c|c|c|c|c|c|c|c|c|}
\multicolumn{17}{c}{$\Delta X$ values}\\\hline
\multirow{16}{*}{$X$}
& 0 & 1 & 2 & 3 & 4 & 5 & 6 & 7 & 8 & 9 & A & B & C & D & E & F \\\hline\hline
0 & 0 & 6 & 1 & 2 & 5 & 4 & 7 & 3 & 9 & F & 8 & B & A & C & E & D \\\hline
1 & 0 & 6 & 4 & 7 & 2 & 3 & 5 & 1 & 9 & F & D & E & A & C & B & 8 \\\hline
2 & 0 & 3 & 1 & 7 & 6 & 2 & 4 & 5 & 9 & A & 8 & E & F & C & B & D \\\hline
3 & 0 & 3 & 4 & 2 & 1 & 5 & 6 & 7 & 9 & A & D & B & F & C & E & 8 \\\hline
4 & 0 & 1 & 2 & 6 & 5 & 3 & 4 & 7 & F & 9 & B & 8 & C & A & D & E \\\hline
5 & 0 & 1 & 7 & 3 & 2 & 4 & 6 & 5 & 8 & E & 9 & A & B & D & F & C \\\hline
6 & 0 & 4 & 2 & 3 & 6 & 5 & 7 & 1 & 9 & A & D & B & F & C & E & 8 \\\hline
7 & 0 & 4 & 7 & 6 & 1 & 2 & 5 & 3 & E & D & F & 9 & 8 & B & C & A \\\hline
8 & 0 & 6 & 1 & 2 & 3 & 5 & 7 & 4 & 9 & F & 8 & B & C & D & E & A \\\hline
9 & 0 & 6 & 4 & 7 & 3 & 5 & 2 & 1 & 9 & F & D & E & B & A & C & 8 \\\hline
A & 0 & 3 & 1 & 7 & 6 & 5 & 2 & 4 & 9 & A & 8 & E & F & B & D & C \\\hline
B & 0 & 3 & 4 & 2 & 6 & 5 & 7 & 1 & 9 & A & D & B & 8 & C & F & E \\\hline
C & 0 & 6 & 4 & 7 & 3 & 5 & 2 & 1 & F & E & D & 9 & A & C & B & 8 \\\hline
D & 0 & 6 & 1 & 2 & 3 & 5 & 7 & 4 & 8 & 9 & F & B & A & C & E & D \\\hline
E & 0 & 3 & 4 & 2 & 6 & 5 & 7 & 1 & 9 & D & B & A & F & C & E & 8 \\\hline
F & 0 & 3 & 1 & 7 & 6 & 5 & 2 & 4 & E & A & 9 & 8 & F & C & B & D \\\hline
\end{tabular}
\end{center}

Each value in the above table is the $\Delta Y$ for a given $X$, the
row label, and $\Delta X$, the column label. To work out the $\Delta Y$
by hand, use the following formula:

\begin{equation}
\Delta Y = S(X) \oplus S(X \oplus \Delta X)
\end{equation}

By counting the number of times a particular output difference $\Delta Y$
occurs for a particular input difference $\Delta X$, we can generate
the following table, where the column headings are output differences
and the row headings are input differences:

\begin{center}
\begin{tabular}{|c||c|c|c|c|c|c|c|c|c|c|c|c|c|c|c|c|c|}
\hline
& 0 & 1 & 2 & 3 & 4 & 5 & 6 & 7 & 8 & 9 & A & B & C & D & E & F \\\hline\hline
0 & 16 & 0 & 0 & 0 & 0 & 0 & 0 & 0 & 0 & 0 & 0 & 0 & 0 & 0 & 0 & 0 \\\hline
1 & 0 & 2 & 0 & 6 & 2 & 0 & 6 & 0 & 0 & 0 & 0 & 0 & 0 & 0 & 0 & 0 \\\hline
2 & 0 & 6 & 2 & 0 & 6 & 0 & 0 & 2 & 0 & 0 & 0 & 0 & 0 & 0 & 0 & 0 \\\hline
3 & 0 & 0 & 6 & 2 & 0 & 0 & 2 & 6 & 0 & 0 & 0 & 0 & 0 & 0 & 0 & 0 \\\hline
4 & 0 & 2 & 2 & 4 & 0 & 2 & 6 & 0 & 0 & 0 & 0 & 0 & 0 & 0 & 0 & 0 \\\hline
5 & 0 & 0 & 2 & 2 & 2 & 10 & 0 & 0 & 0 & 0 & 0 & 0 & 0 & 0 & 0 & 0 \\\hline
6 & 0 & 0 & 4 & 0 & 2 & 2 & 2 & 6 & 0 & 0 & 0 & 0 & 0 & 0 & 0 & 0 \\\hline
7 & 0 & 6 & 0 & 2 & 4 & 2 & 0 & 2 & 0 & 0 & 0 & 0 & 0 & 0 & 0 & 0 \\\hline
8 & 0 & 0 & 0 & 0 & 0 & 0 & 0 & 0 & 2 & 10 & 0 & 0 & 0 & 0 & 2 & 2 \\\hline
9 & 0 & 0 & 0 & 0 & 0 & 0 & 0 & 0 & 0 & 2 & 6 & 0 & 0 & 2 & 2 & 4 \\\hline
A & 0 & 0 & 0 & 0 & 0 & 0 & 0 & 0 & 4 & 2 & 0 & 2 & 0 & 6 & 0 & 2 \\\hline
B & 0 & 0 & 0 & 0 & 0 & 0 & 0 & 0 & 2 & 2 & 2 & 6 & 0 & 0 & 4 & 0 \\\hline
C & 0 & 0 & 0 & 0 & 0 & 0 & 0 & 0 & 2 & 0 & 4 & 2 & 2 & 0 & 0 & 6 \\\hline
D & 0 & 0 & 0 & 0 & 0 & 0 & 0 & 0 & 0 & 0 & 2 & 2 & 10 & 2 & 0 & 0 \\\hline
E & 0 & 0 & 0 & 0 & 0 & 0 & 0 & 0 & 0 & 0 & 0 & 4 & 2 & 2 & 6 & 2 \\\hline
F & 0 & 0 & 0 & 0 & 0 & 0 & 0 & 0 & 6 & 0 & 2 & 0 & 2 & 4 & 2 & 0 \\\hline
\end{tabular}
\end{center}

Taking a look at the table above, we see that certain input differences
map to certain output differences significantly more often than others.
For example, the input difference 5 goes to the output difference 5,
10 out of the 16 times, giving us a probability of $\frac{10}{16}$.
For a perfect S-box, we want each input difference go to each output
difference $\frac{1}{2^n} = \frac{1}{16}$. Unfortunately, we know that
$\Delta Y = S(X) \oplus S(X \oplus \Delta X) = S(X \oplus \Delta X) 
\oplus S(X)$ so input differences map to output differences in pairs,
and thus the smallest number we could get is 2.

But we can do this process for each round, and chain together
high-probability differentials between rounds, giving us a 
\textbf{differential trail}.


\section{Statistical Analysis on Differentials}

\section{Breaking Each Round Key}

\section{Combining it all together}

\section{The Theory Behind It or Why It Works}


\begin{comment}
In this chapter we establish some conventions and collect the
results from number theory about $\Z_{p^r}^*$, the multiplicative
group of integers modulo a prime power $p^r$, that we will need in
the rest of the thesis.

\section{Definitions and notation}

We denote the field of rational numbers by $\Q$, the ring of
integers by $\Z$, and the ring of integers modulo an integer $m$
by $\Z_m$.

Let $x$ be a rational number written in lowest terms, i.e., $x =
\frac{a}{b}$ where $a$ and $b$ are coprime integers. When we refer
to ``the numerator" or ``the denominator" of $x$ we will always
mean $a$ and $b$.
% i.e., the denominator of $x$ when it is written in lowest terms.
If $m$ is an integer coprime to $b$ then by $x \bmod m$ we mean $a \, b^{-1} \bmod m$.
The notation $m \divides x$ means that $a \, b^{-1} \equiv 0 \bmod m$, and $m \notdivides
x$ means that $a \, b^{-1} \nequiv 0 \bmod m$.
% Not sure if I ever used this notation for rational numbers.
If $p$ is a prime and $x$ an integer then the notation $p^r
\maxdivides x$ means that $p^r$ is the highest power of $p$
dividing $x$.
% $x$ is divisible by $p^r$ but not by $p^{r+1}$.

If $R$ is a ring then $R^*$ denotes the multiplicative group of
invertible elements of $R$, and $R^+$ the additive group of $R$.
If $K$ is a field then we denote its algebraic closure by
$\overline{K}$.

Finally, we denote the cyclic group with $n$ elements by $C_n$.
If $g$ is an element of a group $G$, then the cyclic subgroup of
$G$ generated by $g$ is $\langle g \rangle$.





\section{The group $\Z_{p^r}^*$}

Let $p^r$ be a prime power.  We need some results about
$\Z_{p^r}^*$, the multiplicative group of invertible integers
modulo $p^r$.  Most of these follow from the fact that
$\Z_{p^r}^*$ is a cyclic group if $p$ is odd, or the product of
two cyclic groups if $p = 2$. (See for example \cite[page
91]{Rose} for a proof.)
% It can be proved that this is a cyclic group if $p$ is odd, or the product of two cyclic
% groups if $p = 2$.

\begin{thm} \label{t:cyclic} % \cite{Rose}  Rose pg 91

Let $r \in \N$.  If $p$ is an odd prime then $\Z_{p^r}^*$ is a cyclic group of order
$p^{r-1} \, (p-1)$.  For $r \ge 2$, $\Z_{2^r}^*$ is the direct product of two cyclic
groups of order $2$ and $2^{r-2}$ respectively.
% $\Z_{2^r}^* \cong C_2 \times C_{2^{r-2}$.  In other words, there exist elements $h$ and
% $g$ of order $2$ and $2^{r-2}$ respectively in $\Z_{2^r}^*$ such that every element $x$
% of $\Z_{2^r}^*$ can be written uniquely as $x \equiv h^s \, g^t \bmod 2^r$ for some
% $s \in \{0,1\}$ and $t \in \{0,1, \ldots, 2^{r-2}-1 \}$.  (i.e., $\Z_{2^r}^*$ has a basis
% of two elements, a generator pair.) Specifically, $h = -1$ will work.

\end{thm}

% So $\phi(p^r) = p^{r-1}(p-1)$ holds for $p = 2$ too.
% Note $Z_{2^r}$ still has $2^{r-1}$ elements.
% Note $2^r \divides x^2 - 1$ \; \Leftrightarrow \; $2^{r-1} \divides x \pm 1.

So for $r \ge 2$, there is an element $h$ of order $2^{r-2}$ in
$\Z_{2^r}^*$ such that each $x \in \Z_{2^r}^*$ can be written
uniquely as $\pm h^s$ for some $s \in \{0,1,\ldots,2^{r-2}\}$.
%as $(-1)^t \, h^s$ for some $s \in \{0,1,\ldots,2^{r-2}\}$ and $t \in \{0,1\}$.
%$(-1,h)$ is called a generating pair.
%(So each element of $\Z_{2^r}^*$ has order dividing $\min\{2,
%2^{r-2}\}$.)
%So $\Z_{2^r}^*$ is not cyclic unless $r = 1$ or $2$.

%Note that if $r \ge 3$ then $h$ is a quadratic non-residue (because
%if $h \equiv w^2 \bmod 2^r$ then $h^{\frac{2^{r-3}} \equiv w^{2^{r-2}} \equiv 1 \bmod 2^r$,
%and $h$ can't have order $2^{r-2}$).
% So for $r \ge 3$, $x \equiv (-1)^t \, h^s \bmod 2^r$ is a QR modulo $2^r$ if and only if $t+s$ is even.
% So exactly half the elements of $\Z_{2^r}^*$ are quadratic residues.  This is also true for $r = 2$.


\bigskip

Theorem \ref{t:cyclic} has several immediate consequences:

% Obviously if $g$ is a generator of $\Z_{p^r}^*$ then it's also a generator of $\Z_{p^{\ell}}^*$
% for any $1 \le \ell \le r$.

\begin{lem}

Let $p$ be an odd prime, $r \in \N$ and $g$ a generator of $\Z_{p^r}^*$.  Then the
element $g^s$ has order
\[ \dfrac{p^{r-1} \, (p-1)} {\gcd \left(p^{r-1} \, (p-1), \, s \right)} \]
in $\Z_{p^r}^*$, and $g^s$ is a quadratic residue modulo $p^r$ if and only if $s$ is
even.

\end{lem}

% So exactly half the elements of $\Z_{p^r}^*$ are quadratic residues.
% This follows from the fact that $g$ is a quadratic non-residue (because
% if $g = w^2$ then $g^{\frac{p^{r-1} \, (p-1)}{2}}
% \equiv w^{p^{r-1} \, (p-1)} \equiv 1 \bmod p^r$,
% and $g$ cannot be a generator).

\begin{lem} \label{l:same generator}

Let $p$ be an odd prime, $r \in \N$ and $g$ a generator of $\Z_{p^r}^*$.  Then $g$ is
also a generator for $\Z_{p^k}^*$ for all $k < r$.

Let $(-1,h)$ be a generating pair for $\Z_{2^r}^*$.  Then it is
also a generating pair for
$\Z_{2^k}^*$ for all $k < r$. % $h$ has order $2^{r-2}$.

%This is really the same as saying if $h$ has order $2^{r-2}$ modulo $2^r$ then it has order $2^{k-2}$ modulo $2^k$
%for $k \le r$.

\end{lem}

%Note $-1 \in \Z_{2^r} \leftrightarrow (-1,1) \in C_2 \times C_{2^{r-2}}$, not $(1,-1)$.
%$-1$ is not a power of $g$.

% $g$ is a generator of $\Z_{2^r}^* / \{-1,1\}$?

\begin{thm} \label{t:square roots of 1}

Let $r \in \N$.  If $p$ is an odd prime, then
\[ x^2 \equiv 1 \bmod p^r \; \Leftrightarrow \; x \equiv \pm 1 \bmod p^r. \]
For $p = 2$ we have
\begin{align*}
x^2 \equiv 1 \bmod 2^r \;
    &\Leftrightarrow \;
        x \equiv \pm 1 \bmod 2^r \text{~~ or ~~} x \equiv \pm 1 + 2^{r-1} \bmod 2^r \\
    &\Leftrightarrow \; x \equiv \pm 1 \bmod 2^{r-1}.
\end{align*}

\end{thm}

% \begin{prf}

% If $p$ is odd let $g$ be a generator of $\Z_{p^r}^*$.  Then
% $g^{2s} \equiv 1 \bmod p^r$ iff $p^{r-1} \, (p-1) \divides 2s$
% iff $\frac{1}{2} p^{r-1} \, (p-1) \divides s$ iff $g^s \equiv \pm 1 \bmod p^r$.

% If $p = 2$ then $x^2 \equiv 1 \bmod p^r \not \Rightarrow x \equiv \pm 1 \bmod p^r$.
% To see this, note that
% \[ x^2 \equiv 1 \bmod 2^r \Leftrightarrow 2^r \divides x^2-1 = (x-1)(x+1), \]
% and $2$ can divide both $(x-1)$ and $(x+1)$. % which is the difference from other primes $p$.
% But $4$ cannot divide both though, so if
% \[ 2^j \divides (x-1) \text{ and } 2^{r-j} \divides (x+1) \]
% then we must have $j = 0,1, r-1$ or $r$.
% So $x \equiv \pm 1 \bmod 2^{r-1}$, i.e., $x = \pm 1 + c 2^{r-1}$ for some integer $c$.
% So modulo $2^r$, $x$ is $\pm 1 + (c \bmod 2) 2^{r-1}$.
% In other words, $x \equiv \pm 1 \bmod 2^r$ or $x \equiv \pm 1 + 2^{r-1} \bmod 2^r$.
% So $1$ has four square roots modulo $2$.

% $-1$ is a QNR modulo $4$, so $y^{2^k}+1$ divisible by $2$ but not by $4$.
% \qed \end{prf}

So $1$ has two square roots in $\Z_{p^r}^*$ if $p$ is an odd
prime, and four square roots in $\Z_{2^r}^*$ for each $r \ge 3$
(namely, $\pm 1$ and $\pm 1 + 2^{r-1}$).
% (For $r = 2$ $\pm 1 \equiv \mp 3 \bmod 4$.)
% i.e., $1$, $h$, $g^{2^{r-3}}$ and $h \, g^{2^{r-3}}$.

% For example, if $x = 7$ and $r = 4$, then the four square roots
% of $1$ modulo $2^4$ are $\pm 1$ and $\pm 1 + 2^3$, i.e., $9$ and $7$.
\end{comment}



%%%%%%%%%%%%%%%%%%%%%%%%%%%%%%%%%%%%%%%%%%%%%%%%%%%%%%%%%%%%%%%%%%%
%                                                                 %
%                       Differential Attacks                      %
%                                                                 %
%%%%%%%%%%%%%%%%%%%%%%%%%%%%%%%%%%%%%%%%%%%%%%%%%%%%%%%%%%%%%%%%%%%


\chapter{Differential Attacks} \label{c:differential attacks}

So far we have taken a look at a number of sections, which for the most part
look fairly disjoint. In this section, we will tie these concepts together and
discuss the process whereby a differential attack can be mounted against a
block cipher. Perhaps the easiest way to do this would be by means of working
through the process on an example in various subsections.

So before we begin, we would want to clarify what assumptions need to be made
in order to mount a successful attack against the type of Block Cipher
discussed in the previous chapter.
\paragraph{}

We can reduce them to the following list:
\begin{itemize}
\item We can choose some plaintext.
\item We know or can generate the corresponding ciphertext.
\item We know the SP-boxes. 
\item We do not know the key.
\end{itemize}

Accepting those assumptions, we are ready to note down our algorithm for
attacking Block Ciphers, and specifically a SPN, with Differential
Cryptanalysis. So let's explain how we would attack an SPN then...

Firstly, for each round of our Iterated Block Cipher, we will want to search
for any pairs of inputs $(X, X')$ with XOR difference $\Delta X$, which results
in the corresponding output pair $(Y, Y')$ having XOR difference $\Delta Y$
with high probability. To be clear however in the above, $X \oplus X' =
\Delta X$ and $Y \oplus Y' = \Delta Y$. The pair $(\Delta X, \Delta Y)$ is
known as a \textbf{differential}. We can do this, since although SP-boxes affect
inputs and outputs, they do not affect differentials. This enables us to to build up a
differential characteristic for the Block Cipher. 

Thinking back to the structure of our SPN, if there is an SP-box after the last
round, we can simply invert it and get what the input would be to said SP-box,
so either way we are left facing the last round key. Then, since we know 
high-probability inputs for the last round from our differential characteristic,
we can XOR the inputs with our know ciphertext to extract the key.

In summary, once we have chosen our plaintext/ciphertext pairs, our basic
methodology for breaking the last round is as follows:
\begin{enumerate}
\item For each round up to $r-1$, compute the differentials on the SP-box.
\item Find differential characteristics with high probabilities.
\item Extract the last round key.
\end{enumerate}

We will use this strategy iteratively to break each round key, so let us
discuss each of these points in greater detail.

\section{Computing Differentials for Each Round}
So, for each round, how do we analyze an SP-box to retrieve its differentials?
Well since the SP-boxes are publicly known, we can iterate through all possible
inputs $X$ to a particular SP-box, and for each input, iterate through all
possible input differences $\Delta X$ to get input pairs $(X, X' = X \oplus
\Delta X$. Then, by sending the inputs, $X$ and $X'$ through the SP-box, we can
calculate each possible output difference $\Delta Y$ given by $\Delta Y = S(X)
\oplus S(X')$. Once we have all of these output differences, we can tally the
output differences that occur often for a given input difference, and record
them in a differential table. 

This might be best explained using an example, so let us consider the SP-box 
represented by the following table (in hexadecimal):
\begin{center}
\begin{tabular}{|r|r|}
\hline
$X$ & $S(X)$ \\\hline
0 & 3 \\\hline
1 & 5 \\\hline
2 & 2 \\\hline
3 & 1 \\\hline
4 & 6 \\\hline
5 & 7 \\\hline
6 & 4 \\\hline
7 & 0 \\\hline
8 & A \\\hline
9 & C \\\hline
A & B \\\hline
B & 8 \\\hline
C & 9 \\\hline
D & F \\\hline
E & D \\\hline
F & E \\\hline
\end{tabular}
\end{center}
You might notice that in this SP-box, function values do not seem to be particularly
well distributed. In fact, any value less than 8 is mapped to corresponding
value less than 8, and consequently values greater than or equal to 8 are mapped
to values greater than 8. In real life, this would correspond to a bad permutation
function, but we purposefully use this SP-box so that we get clear outliers in terms of
high-probabities differentials when we compute the differential table.

Thus, if we step through possible inputs, $\{0, ..., 15\}$ to our SP-box, and for
each input, XOR it with all the possible input differences $\{0, ..., 15\}$, we
get the following pairs $(X, X \oplus \Delta X)$ represented by the row and
column headings of the table below:
\begin{center}
\begin{tabular}{|c||c|c|c|c|c|c|c|c|c|c|c|c|c|c|c|c|c|}
%\multicolumn{17}{c}{$\Delta X$ values}\\
\hline
%\multirow{16}{*}{$X$}
& 0 & 1 & 2 & 3 & 4 & 5 & 6 & 7 & 8 & 9 & A & B & C & D & E & F \\\hline\hline
0 & 0 & 6 & 1 & 2 & 5 & 4 & 7 & 3 & 9 & F & 8 & B & A & C & E & D \\\hline
1 & 0 & 6 & 4 & 7 & 2 & 3 & 5 & 1 & 9 & F & D & E & A & C & B & 8 \\\hline
2 & 0 & 3 & 1 & 7 & 6 & 2 & 4 & 5 & 9 & A & 8 & E & F & C & B & D \\\hline
3 & 0 & 3 & 4 & 2 & 1 & 5 & 6 & 7 & 9 & A & D & B & F & C & E & 8 \\\hline
4 & 0 & 1 & 2 & 6 & 5 & 3 & 4 & 7 & F & 9 & B & 8 & C & A & D & E \\\hline
5 & 0 & 1 & 7 & 3 & 2 & 4 & 6 & 5 & 8 & E & 9 & A & B & D & F & C \\\hline
6 & 0 & 4 & 2 & 3 & 6 & 5 & 7 & 1 & 9 & A & D & B & F & C & E & 8 \\\hline
7 & 0 & 4 & 7 & 6 & 1 & 2 & 5 & 3 & E & D & F & 9 & 8 & B & C & A \\\hline
8 & 0 & 6 & 1 & 2 & 3 & 5 & 7 & 4 & 9 & F & 8 & B & C & D & E & A \\\hline
9 & 0 & 6 & 4 & 7 & 3 & 5 & 2 & 1 & 9 & F & D & E & B & A & C & 8 \\\hline
A & 0 & 3 & 1 & 7 & 6 & 5 & 2 & 4 & 9 & A & 8 & E & F & B & D & C \\\hline
B & 0 & 3 & 4 & 2 & 6 & 5 & 7 & 1 & 9 & A & D & B & 8 & C & F & E \\\hline
C & 0 & 6 & 4 & 7 & 3 & 5 & 2 & 1 & F & E & D & 9 & A & C & B & 8 \\\hline
D & 0 & 6 & 1 & 2 & 3 & 5 & 7 & 4 & 8 & 9 & F & B & A & C & E & D \\\hline
E & 0 & 3 & 4 & 2 & 6 & 5 & 7 & 1 & 9 & D & B & A & F & C & E & 8 \\\hline
F & 0 & 3 & 1 & 7 & 6 & 5 & 2 & 4 & E & A & 9 & 8 & F & C & B & D \\\hline
\end{tabular}
\end{center}

Each value in the above table is the $\Delta Y$ for a given $X$, the
row label, and $\Delta X$, the column label. To work out the $\Delta Y$
by hand, use the following formula:

\begin{equation}
\Delta Y = S(X) \oplus S(X \oplus \Delta X)
\end{equation}

By counting the number of times a particular output difference $\Delta Y$
occurs for a particular input difference $\Delta X$, we can generate
the following table, where the column headings are output differences
and the row headings are input differences:

\begin{center}
\begin{tabular}{|c||c|c|c|c|c|c|c|c|c|c|c|c|c|c|c|c|c|}
\hline
& 0 & 1 & 2 & 3 & 4 & 5 & 6 & 7 & 8 & 9 & A & B & C & D & E & F \\\hline\hline
0 & 16 & 0 & 0 & 0 & 0 & 0 & 0 & 0 & 0 & 0 & 0 & 0 & 0 & 0 & 0 & 0 \\\hline
1 & 0 & 2 & 0 & 6 & 2 & 0 & 6 & 0 & 0 & 0 & 0 & 0 & 0 & 0 & 0 & 0 \\\hline
2 & 0 & 6 & 2 & 0 & 6 & 0 & 0 & 2 & 0 & 0 & 0 & 0 & 0 & 0 & 0 & 0 \\\hline
3 & 0 & 0 & 6 & 2 & 0 & 0 & 2 & 6 & 0 & 0 & 0 & 0 & 0 & 0 & 0 & 0 \\\hline
4 & 0 & 2 & 2 & 4 & 0 & 2 & 6 & 0 & 0 & 0 & 0 & 0 & 0 & 0 & 0 & 0 \\\hline
5 & 0 & 0 & 2 & 2 & 2 & 10 & 0 & 0 & 0 & 0 & 0 & 0 & 0 & 0 & 0 & 0 \\\hline
6 & 0 & 0 & 4 & 0 & 2 & 2 & 2 & 6 & 0 & 0 & 0 & 0 & 0 & 0 & 0 & 0 \\\hline
7 & 0 & 6 & 0 & 2 & 4 & 2 & 0 & 2 & 0 & 0 & 0 & 0 & 0 & 0 & 0 & 0 \\\hline
8 & 0 & 0 & 0 & 0 & 0 & 0 & 0 & 0 & 2 & 10 & 0 & 0 & 0 & 0 & 2 & 2 \\\hline
9 & 0 & 0 & 0 & 0 & 0 & 0 & 0 & 0 & 0 & 2 & 6 & 0 & 0 & 2 & 2 & 4 \\\hline
A & 0 & 0 & 0 & 0 & 0 & 0 & 0 & 0 & 4 & 2 & 0 & 2 & 0 & 6 & 0 & 2 \\\hline
B & 0 & 0 & 0 & 0 & 0 & 0 & 0 & 0 & 2 & 2 & 2 & 6 & 0 & 0 & 4 & 0 \\\hline
C & 0 & 0 & 0 & 0 & 0 & 0 & 0 & 0 & 2 & 0 & 4 & 2 & 2 & 0 & 0 & 6 \\\hline
D & 0 & 0 & 0 & 0 & 0 & 0 & 0 & 0 & 0 & 0 & 2 & 2 & 10 & 2 & 0 & 0 \\\hline
E & 0 & 0 & 0 & 0 & 0 & 0 & 0 & 0 & 0 & 0 & 0 & 4 & 2 & 2 & 6 & 2 \\\hline
F & 0 & 0 & 0 & 0 & 0 & 0 & 0 & 0 & 6 & 0 & 2 & 0 & 2 & 4 & 2 & 0 \\\hline
\end{tabular}
\end{center}

Taking a look at the table above, we see that certain input differences map to
certain output differences significantly more often than others.  For example,
the input difference 5 goes to the output difference 5, 10 out of the 16 times,
giving us a probability of $\frac{10}{16}$.  For a perfect SP-box, we want each
input difference go to each output difference $\frac{1}{2^n} = \frac{1}{16}$.
Unfortunately, we know that $\Delta Y = S(X) \oplus S(X \oplus \Delta X) = S(X
\oplus \Delta X) \oplus S(X)$ so input differences map to output differences in
pairs, and thus the smallest number we could get is 2.

Up until now, we have all but ignored the effect of the key, but there is a good
reason why: the key does not affect the differentials of a round.

To see this, notice that for inputs $X$ and $X'$ and round subkey $K$, we have:
\begin{align*}
(X \oplus K) \oplus (X' \oplus K) &= X \oplus K \oplus X' \oplus K\\
&= X \oplus X' \oplus K \oplus K\\
&= X \oplus X'\\
&= \Delta X
\end{align*}

\section{Constructing a Differential Characteristic}

In the previous section, we saw how to calculate the differentials for a
particular round. However, we can conduct this process for each round, and
chain together high-probability differentials between rounds, giving us a
\textbf{differential characteristic} of the Block Cipher. We do this by using
one round's high probability output differential as the next round's input
differential.

\begin{example}
Continuing the example from the previous section, we saw that the following
input differences went to the following output differences with high
probability:
\begin{itemize}
\item $5 \rightarrow 5$ with probability $\frac{10}{16}$
\item $8 \rightarrow 9$ with probability $\frac{10}{16}$
\item $D \rightarrow C$ with probability $\frac{10}{16}$
\end{itemize}
Now suppose this was our first SP-box, with our SPN consisting of 4 rounds, we
would have the same information for each of these rounds. Perhaps the second
SP-box had a differential $9 \rightarrow 2$ with probability $\frac{8}{16}$
and the third SP-box had a differential $2 \rightarrow 6$ with probability 
$\frac{6}{16}$. Then, since these probabilities are assumed to be independent,
we can multiply them together to get a differential characteristic $8 \rightarrow
2$ with probability $\frac{10}{16} \times \frac{8}{16} \times \frac{10}{16} = 
\frac{600}{4096} = \frac{75}{512}$ for the first 3 rounds. Thus, we are assured
that plaintext pairs with a difference of 8 will result in input pairs to round
4 with a difference of 2 with high probability.
\end{example}

\begin{rem}
If the probabilities are not independent, then we can't multiply them together
and the differential attack would not work.
\end{rem}

We then use this differential charateristic in the next stage of our attack.

\section{Extracting Each Round Key}
How would we extract a key from a round? Well if we knew the input $X$ to a round,
the SP-box and the output $Y$, we could send the output back through the SP-box, to
get $S^{-1}(Y) = X \oplus K $ and XOR it with $X$ to get the key K since
\begin{equation}
S(X \oplus K) = Y.
\end{equation}

For the last round, we know what the outputs are since we can generate ciphertext
from plaintext. Unfortunately, we do not know the specific input to our last round
before the key gets mixed in. We do however know what possible plaintext input 
differences will lead to what possible input differences to round $r$ from our 
differential characteristic.

All that is then required is to choose plaintext pairs $(P, P')$ satisfying our
high-probability differential characteristic, and generate corresponding inputs
to round $r$ from the input difference to round $r$.  Now for each input pair
$(X, X')$, we can XOR the inputs $X$ and $X'$ with the inputs to our SP-box
$S^{-1}(C)$ and $S^{-1}(C')$, such that we get:

\begin{equation}
X \oplus S^{-1}(C) = K
\end{equation}
\begin{equation}
X' \oplus S^{-1}(C') = K'
\end{equation}

Thus, we get possible values for the round key $K$ and $K'$. If we have chosen
a high probability differential on which we base our input pairs, there should
be a $K$ that clearly occurs more often than others. This is then our candidate
key for that round. We then check this key against arbitrary plaintext inputs
to ensure that we get the expected ciphertext outputs, and once the key is
verified, we accept as being the correct key for the round.

Now we repeat the method for our block cipher consisting of $r-1$ rounds, and
extract each round key. Once we have all the round keys, we can easily generate
the key and we will then have broken the SPN.

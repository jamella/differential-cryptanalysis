

%%%%%%%%%%%%%%%%%%%%%%%%%%%%%%%%%%%%%%%%%%%%%%%%%%%%%%%%%%%%%%%%%%%
%                                                                 %
%                           Background                            %
%                                                                 %
%%%%%%%%%%%%%%%%%%%%%%%%%%%%%%%%%%%%%%%%%%%%%%%%%%%%%%%%%%%%%%%%%%%


\chapter{Background} \label{c:background}

Most mathematical concepts require a bit of supporting or background
knowledge, and Cryptography is no different. We will start by looking
at some basic Probability Theory, and then move on to Block Ciphers
and interesting properties surrounding them. Since the amount of
Boolean Algebra required to understand this paper is minimal, a short
comment on this topic will be included in the Block Ciphers section.

\begin{comment}
In this chapter we establish some conventions and collect the
results from number theory about $\Z_{p^r}^*$, the multiplicative
group of integers modulo a prime power $p^r$, that we will need in
the rest of the thesis.
\end{comment}

\section{Probability Theory}
By 3rd year level however, a student will not have covered enough
Probability Theory to understand what the premise behind a 
differential attack is; which means that Probability Theory will
be discussed in this section. Topics include what it means for
and event to have a probability of occuring, how we calculate 
an event's probability of occuring, and how likely an
event is, or string of random events are to occur.


\section{Block Ciphers}
We will be considering Attacks on Block Ciphers later, and thus
it makes sense to introduce Block Ciphers as part of the background.
In short, Block Ciphers will be defined as algorithms that operate
on a fixed amount of bits, using some sort of symmetric key. They
will further be explained and examples will be given.
 
\subsection{Vectorial Boolean Functions (S-boxes)}
A large part of understanding Block Ciphers, and how to attack them,
will be tied up in understanding S-boxes, (which are types of 
Vectorial Boolean Functions). This section will deal with introducing
and explaining them, along with a few examples.

\section{Differential Properties of S-boxes}
Next will be discussed the various differential properties of S-boxes
which allow cryptanalysts to mount an attack against a block cipher. 
These include the property that XORs do not affect differentials, as well
as ways to find relationships between input differentials and output
differentials of an S-box.

\subsection{Probabilities of Differential Trails}
The previous properties discussed can be combined to give us insight into
the probabilites of differentials trails, or in plain English: Given an
input differential, how likely or probable is it that a certain output
differential occurs. This section will discuss the theory behind probabilites
of differentials, as well as practical examples of cases where probabilities
do not conform to the norm.

\begin{comment}
\section{Definitions and notation}

We denote the field of rational numbers by $\Q$, the ring of
integers by $\Z$, and the ring of integers modulo an integer $m$
by $\Z_m$.

Let $x$ be a rational number written in lowest terms, i.e., $x =
\frac{a}{b}$ where $a$ and $b$ are coprime integers. When we refer
to ``the numerator" or ``the denominator" of $x$ we will always
mean $a$ and $b$.
% i.e., the denominator of $x$ when it is written in lowest terms.
If $m$ is an integer coprime to $b$ then by $x \bmod m$ we mean $a \, b^{-1} \bmod m$.
The notation $m \divides x$ means that $a \, b^{-1} \equiv 0 \bmod m$, and $m \notdivides
x$ means that $a \, b^{-1} \nequiv 0 \bmod m$.
% Not sure if I ever used this notation for rational numbers.
If $p$ is a prime and $x$ an integer then the notation $p^r
\maxdivides x$ means that $p^r$ is the highest power of $p$
dividing $x$.
% $x$ is divisible by $p^r$ but not by $p^{r+1}$.

If $R$ is a ring then $R^*$ denotes the multiplicative group of
invertible elements of $R$, and $R^+$ the additive group of $R$.
If $K$ is a field then we denote its algebraic closure by
$\overline{K}$.

Finally, we denote the cyclic group with $n$ elements by $C_n$.
If $g$ is an element of a group $G$, then the cyclic subgroup of
$G$ generated by $g$ is $\langle g \rangle$.





\section{The group $\Z_{p^r}^*$}

Let $p^r$ be a prime power.  We need some results about
$\Z_{p^r}^*$, the multiplicative group of invertible integers
modulo $p^r$.  Most of these follow from the fact that
$\Z_{p^r}^*$ is a cyclic group if $p$ is odd, or the product of
two cyclic groups if $p = 2$. (See for example \cite[page
91]{Rose} for a proof.)
% It can be proved that this is a cyclic group if $p$ is odd, or the product of two cyclic
% groups if $p = 2$.

\begin{thm} \label{t:cyclic} % \cite{Rose}  Rose pg 91

Let $r \in \N$.  If $p$ is an odd prime then $\Z_{p^r}^*$ is a cyclic group of order
$p^{r-1} \, (p-1)$.  For $r \ge 2$, $\Z_{2^r}^*$ is the direct product of two cyclic
groups of order $2$ and $2^{r-2}$ respectively.
% $\Z_{2^r}^* \cong C_2 \times C_{2^{r-2}$.  In other words, there exist elements $h$ and
% $g$ of order $2$ and $2^{r-2}$ respectively in $\Z_{2^r}^*$ such that every element $x$
% of $\Z_{2^r}^*$ can be written uniquely as $x \equiv h^s \, g^t \bmod 2^r$ for some
% $s \in \{0,1\}$ and $t \in \{0,1, \ldots, 2^{r-2}-1 \}$.  (i.e., $\Z_{2^r}^*$ has a basis
% of two elements, a generator pair.) Specifically, $h = -1$ will work.

\end{thm}

% So $\phi(p^r) = p^{r-1}(p-1)$ holds for $p = 2$ too.
% Note $Z_{2^r}$ still has $2^{r-1}$ elements.
% Note $2^r \divides x^2 - 1$ \; \Leftrightarrow \; $2^{r-1} \divides x \pm 1.

So for $r \ge 2$, there is an element $h$ of order $2^{r-2}$ in
$\Z_{2^r}^*$ such that each $x \in \Z_{2^r}^*$ can be written
uniquely as $\pm h^s$ for some $s \in \{0,1,\ldots,2^{r-2}\}$.
%as $(-1)^t \, h^s$ for some $s \in \{0,1,\ldots,2^{r-2}\}$ and $t \in \{0,1\}$.
%$(-1,h)$ is called a generating pair.
%(So each element of $\Z_{2^r}^*$ has order dividing $\min\{2,
%2^{r-2}\}$.)
%So $\Z_{2^r}^*$ is not cyclic unless $r = 1$ or $2$.

%Note that if $r \ge 3$ then $h$ is a quadratic non-residue (because
%if $h \equiv w^2 \bmod 2^r$ then $h^{\frac{2^{r-3}} \equiv w^{2^{r-2}} \equiv 1 \bmod 2^r$,
%and $h$ can't have order $2^{r-2}$).
% So for $r \ge 3$, $x \equiv (-1)^t \, h^s \bmod 2^r$ is a QR modulo $2^r$ if and only if $t+s$ is even.
% So exactly half the elements of $\Z_{2^r}^*$ are quadratic residues.  This is also true for $r = 2$.


\bigskip

Theorem \ref{t:cyclic} has several immediate consequences:

% Obviously if $g$ is a generator of $\Z_{p^r}^*$ then it's also a generator of $\Z_{p^{\ell}}^*$
% for any $1 \le \ell \le r$.

\begin{lem}

Let $p$ be an odd prime, $r \in \N$ and $g$ a generator of $\Z_{p^r}^*$.  Then the
element $g^s$ has order
\[ \dfrac{p^{r-1} \, (p-1)} {\gcd \left(p^{r-1} \, (p-1), \, s \right)} \]
in $\Z_{p^r}^*$, and $g^s$ is a quadratic residue modulo $p^r$ if and only if $s$ is
even.

\end{lem}

% So exactly half the elements of $\Z_{p^r}^*$ are quadratic residues.
% This follows from the fact that $g$ is a quadratic non-residue (because
% if $g = w^2$ then $g^{\frac{p^{r-1} \, (p-1)}{2}}
% \equiv w^{p^{r-1} \, (p-1)} \equiv 1 \bmod p^r$,
% and $g$ cannot be a generator).

\begin{lem} \label{l:same generator}

Let $p$ be an odd prime, $r \in \N$ and $g$ a generator of $\Z_{p^r}^*$.  Then $g$ is
also a generator for $\Z_{p^k}^*$ for all $k < r$.

Let $(-1,h)$ be a generating pair for $\Z_{2^r}^*$.  Then it is
also a generating pair for
$\Z_{2^k}^*$ for all $k < r$. % $h$ has order $2^{r-2}$.

%This is really the same as saying if $h$ has order $2^{r-2}$ modulo $2^r$ then it has order $2^{k-2}$ modulo $2^k$
%for $k \le r$.

\end{lem}

%Note $-1 \in \Z_{2^r} \leftrightarrow (-1,1) \in C_2 \times C_{2^{r-2}}$, not $(1,-1)$.
%$-1$ is not a power of $g$.

% $g$ is a generator of $\Z_{2^r}^* / \{-1,1\}$?

\begin{thm} \label{t:square roots of 1}

Let $r \in \N$.  If $p$ is an odd prime, then
\[ x^2 \equiv 1 \bmod p^r \; \Leftrightarrow \; x \equiv \pm 1 \bmod p^r. \]
For $p = 2$ we have
\begin{align*}
x^2 \equiv 1 \bmod 2^r \;
    &\Leftrightarrow \;
        x \equiv \pm 1 \bmod 2^r \text{~~ or ~~} x \equiv \pm 1 + 2^{r-1} \bmod 2^r \\
    &\Leftrightarrow \; x \equiv \pm 1 \bmod 2^{r-1}.
\end{align*}

\end{thm}

% \begin{prf}

% If $p$ is odd let $g$ be a generator of $\Z_{p^r}^*$.  Then
% $g^{2s} \equiv 1 \bmod p^r$ iff $p^{r-1} \, (p-1) \divides 2s$
% iff $\frac{1}{2} p^{r-1} \, (p-1) \divides s$ iff $g^s \equiv \pm 1 \bmod p^r$.

% If $p = 2$ then $x^2 \equiv 1 \bmod p^r \not \Rightarrow x \equiv \pm 1 \bmod p^r$.
% To see this, note that
% \[ x^2 \equiv 1 \bmod 2^r \Leftrightarrow 2^r \divides x^2-1 = (x-1)(x+1), \]
% and $2$ can divide both $(x-1)$ and $(x+1)$. % which is the difference from other primes $p$.
% But $4$ cannot divide both though, so if
% \[ 2^j \divides (x-1) \text{ and } 2^{r-j} \divides (x+1) \]
% then we must have $j = 0,1, r-1$ or $r$.
% So $x \equiv \pm 1 \bmod 2^{r-1}$, i.e., $x = \pm 1 + c 2^{r-1}$ for some integer $c$.
% So modulo $2^r$, $x$ is $\pm 1 + (c \bmod 2) 2^{r-1}$.
% In other words, $x \equiv \pm 1 \bmod 2^r$ or $x \equiv \pm 1 + 2^{r-1} \bmod 2^r$.
% So $1$ has four square roots modulo $2$.

% $-1$ is a QNR modulo $4$, so $y^{2^k}+1$ divisible by $2$ but not by $4$.
% \qed \end{prf}

So $1$ has two square roots in $\Z_{p^r}^*$ if $p$ is an odd
prime, and four square roots in $\Z_{2^r}^*$ for each $r \ge 3$
(namely, $\pm 1$ and $\pm 1 + 2^{r-1}$).
% (For $r = 2$ $\pm 1 \equiv \mp 3 \bmod 4$.)
% i.e., $1$, $h$, $g^{2^{r-3}}$ and $h \, g^{2^{r-3}}$.

% For example, if $x = 7$ and $r = 4$, then the four square roots
% of $1$ modulo $2^4$ are $\pm 1$ and $\pm 1 + 2^3$, i.e., $9$ and $7$.
\end{comment}

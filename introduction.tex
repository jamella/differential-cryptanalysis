

%%%%%%%%%%%%%%%%%%%%%%%%%%%%%%%%%%%%%%%%%%%%%%%%%%%%%%%%%%%%%%%%%%%
%                                                                 %
%                       Introduction                              %
%                                                                 %
%%%%%%%%%%%%%%%%%%%%%%%%%%%%%%%%%%%%%%%%%%%%%%%%%%%%%%%%%%%%%%%%%%%


\chapter{Introduction} \label{c:introduction}
Imagine a world without encryption. Anyone with access to a computer could
steal your money, impersonate you, or learn your secrets.  Privacy would be
dead. The inability to secure information might even lead to police states,
where governments will have to resort to desperate measures to protect their
secrets.  Fortunately, we do not live in such a world. We live in a world where
encryption is a very real part of our daily lives. Whether banking online,
shopping for new items or chatting to your friends on your favourite social
network, you are indirectly using various forms of encryption to keep your
messages safe, and indicate to computer servers that you are who you say you
are.

So if our data is all secure and encrypted, why should we worry more about the
subject of Cryptography? In short, our data isn't secure. Attackers find more
and more ingenious ways of breaking implementations and protocols, even when
the encryption method is said to be secure. The premier algorithm for
encrypting electronic data from 1979 onwards, known as the Data Encryption
Standard (DES), was publicly broken in 1997 \cite[]{interhack}. Today's
encryption standards are tomorrow's broken algorithms. So we, as
cryptographers, have the responsibility to make encryption more secure; even in
some cases, to improve the manner in which these algorithms are implemented!
How can we do this? By donning a black hat and thinking like an attacker. In
this paper, we will be looking at an attack on block ciphers, specifically
known as differential cryptanalysis. But first, we should probably define some
unfamiliar terminology.

\section{Terminology}

\begin{rem}
Note, this paper assumes that you have an adequate knowledge of 3rd year
university level Mathematics, but a knowledge of Cryptography or Probability
theory is not required, as all concepts necessary for understanding this paper
will be explained.
\end{rem}

As you might already know, \textbf{Cryptography} is the discipline concerned
with keeping data secret, or more formally, it is the study and practice of
techniques and algorithms for securing the transference of data in the presence
of third parties, known as attackers or adversaries.  \textbf{Cryptanalysis}
however, deals with the breaking of these techniques and retrieval of the
secret data. 

In cryptography, \textbf{encryption} refers to the process of converting
plaintext, or data that is easily understandable, into ciphertext, that is
unintelligible data. The reverse process of converting the ciphertext back
into plaintext is known as \textbf{decryption}. In most cases, this encryption
or decryption occurs with the aid of a \textbf{key}, a parameter that
determines the functional output of the cryptographic algorithm.

Furthermore, there are two main types which come up when discussing key-based
cryptography, namely Symmetric-key, and Asymmetric-key or Public-key
cryptography.  In the case of \textbf{Symmetric-key}, the same key is used for
encryption and decryption, while with \textbf{Asymmetric-key}, a key is made
available to the public for encryption, and only those possessing the secret or
private key will be able to decrypt messages encrypted with the matching public
key.

A \textbf{block cipher} is a type of Symmetric-key encryption cipher that
operates on a fixed-length ``block'' of data. This is in contrast to a
\textbf{stream cipher}, which operates on a potentially infinite stream of
data. For the purposes of this paper, we will define a block cipher in more
depth later, but at the moment we have enough of a vocabulary to delve into a
bit of supporting knowledge.


\chapter{LaTeX}

Note: any line that starts with ``\%'' is a comment and will be
ignored by LaTeX.

\section{Math mode}

For numbered equations (you can make the label anything you want
-- I recommend starting it with ``e'' for equation):
\begin{equation} \label{e:quadratic}
y = x^2.
\end{equation}
For unnumbered equations:
\[ y = x^2 \; \; \; \text{ for } x > 0. \]
For equation arrays:
\begin{align}
y &= (x+1)^2 \nonumber \\
  &= x^2 + 2x +1.
\end{align}
* makes it unnumbered:
\begin{align*}
y &= (x+1)^2 \nonumber \\
  &= x^2 + 2x +1.
\end{align*}


\section{Congruences etc}

Look at the next chapter, which I cut and pasted from my thesis.
If there's more stuff you don't know how to write in LaTeX just
email me about it.

\section{General notes}

The command for ``divides" is $\mid$ and the command for ``does
not divide" is $\nmid$.  To write ``is congruent'' to put $\equiv$
and for ``is not congruent to'' us $\not \equiv$.

\smallskip


\section{Style}

\emph{This is how we do italics}.

In Blum's paper \cite{Blum1}...

If you want maths in the middle of the text you put it in dollar
signs, like this: $x = 0$.

Theorems:

\begin{thm} \label{t:regular primes}

Let $(h_n)$ be an EDS and $p$ a prime not dividing $h_2$ or $h_3$.
Then there exists a positive integer $N$ such that
\[ h_n \equiv 0 \pmod p \; \; \Leftrightarrow \; \; n \equiv 0 \pmod N. \]

\end{thm}


\smallskip

This is how we do fractions:  $\frac{a}{b}$.  If you have a
fraction in a displayed equation use
\[ \dfrac{a}{b}. \]

\section{The bibliography}

I put a few references into the file ``Thyla.bib'' for you so you
could see how to do it.  Note that Rose and Blum appear in the
bibliography because they were referenced in the text, but Bach
and Shallit appears because I specifically told it to when I said
``nocite{BachandShallit}'' in the main file Thyla.tex.

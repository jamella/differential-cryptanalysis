


%%%%%%%%%%%%%%%%%%%%%%%%%%%%%%%%%%%%%%%%%%%%%%%%%%%%%%%%%%%%%%%%%%%
%                                                                 %
%                          Protocol                               %
%                                                                 %
%%%%%%%%%%%%%%%%%%%%%%%%%%%%%%%%%%%%%%%%%%%%%%%%%%%%%%%%%%%%%%%%%%%

\chapter{The Protocol}

1. Bob selects $n$\\
Bob chooses, at random, two distint 80-digit primes $p$ and $q$ such that both $p$ and $q \equiv 3 \bmod 4$. Bob multiplies $p$ and $n$ together to obtain $n$ ($n = pq$). Bob sends $n$ to Alice. \\ 
%have to decide where to include publication date of n and the correctness of messages.
\\
2. Alice tests $n$\\
If Alice trusts that Bob has chosen $p$ and $q$ such that they are both prime and both congruent to $3 \bmod 4$, then this part of the protocol may be skipped. However, if not, then Alice checks the following:\\
\\
(i) That $n$ is a 160-digit number and that $n \equiv 1\bmod 4$. This will verify that $n$ is odd and that $\left(\frac{-1}{n}\right) = 1$. The latter statement implies that either both $p$ and $q \equiv 3 \bmod 4$ or that both $p$ and $q \equiv 1 \bmod 4$, $(v)$ Theorem 2.0.4\\
\\
(ii) Alice checks that for some $c$ there is a $d$ such that $c^2 \equiv d^2 \bmod n$ and that $\left(\frac{c}{n}\right) \neq \left(\frac{d}{n}\right)$. She checks this as follows:\\
\\
\indent a) Bob chooses 80 distinct numbers, $c_1,.....,c_{80}$ at random from ${\Z_{n}}^*$. He\\ 
\indent sends ${c_1}^2 \bmod n,.....,{c_{80}}^2 \bmod n$ to Alice.\\
\\
\indent b) Alice sends Bob a sequence of 80 randomly selected bits, $b_1,...,b_{80}$,\\
\indent where $b_i = 1$ or -1.\\
\\
\indent c) Let ${c_i}^2 \equiv {d_i}^2 \bmod n$ where $\left(\frac{c_i}{n}\right)= 1$  and $\left(\frac{d_i}{n}\right) = -1$. Two such roots\\
\indent exist by $(3)$ of Theorem 2.0.8. Bob sends Alice a sequence of 80 numbers;\\
\indent he sends $c_i$ if $b_i = 1$, or $d_i$ if $b_i = -1$. (Blum [1]).\\
\\
(a) through (c) convince Alice that $(1)$ of Theorem 2.0.8 holds. This implies that $(2)$ of Theorem 2.0.8 holds, and in conjunction with (i) above, implies that $p$ and $q$ have to be congruent to $3 \bmod 4$. (Alice does not know the value of $p$ and $q$).\\
\\
3. Alice sets up 'the coin'\\
Alice picks, at random, $a$ such that $a \in {\Z_{n}}^*$. Alice calculates $y \equiv a^2 \bmod n$ and sends $y$ to Bob. At this point, Alice also calculates $\left(\frac{a}{n}\right)$. Alice does not disclose this to Bob.\\
\\
4. Bob picks: heads or tails\\
By $(iii)$ of Theorem (nb one)$a^2 \bmod n$ has as many roots with $\left(\frac{a}{n}\right) = 1$ as it has $\left(\frac{a}{n}\right) = -1$. Bob guesses the value $\left(\frac{a}{n}\right) = 1$. Bob sends his guess to Alice.\\
\\
5. Somebody wins, somebody loses\\
Alice tells Bob whether or not his guess is correct and sends him her initial choice of $a$.\\
\\
The final two steps verify that neither party has cheated:\\
\\
6. Did Alice cheat?\\
Bob checks that the $y$ sent to him by Alice is indeed congruent to $a^2 \bmod n$. He calculates $\left(\frac{a}{n}\right)$ to ensure that Alice was telling the truth in step (4).\\
\\
7. Did Bob cheat?\\
Bob sends $p$ and $q$ to Alice. Although Alice has tested $n$ in step (2), she now knows the value of $p$ and $q$ and verifies that $n = pq$ with both primes congruent to $3 \bmod 4$.\\ 
\\
A random number $d$ between $0$ and $2^r - 1$ can be generated by repeating steps 3 through 6 of the protocol $r$ times. The binary representation of $d$ is obtained by recording $1$ when Bob guesses correctly and $0$ when he guesses incorrectly. As has been mentioned previously, Bob is guaranteed that Alice will pick a sequence of bits at random and Alice is assured that Bob will not know what sequence of bits has been flipped to Alice.\\  
\\
A quick note: If Alice and Bob wish to continue flipping coins to each other, the creation of new primes $p$ and $q$ such that a new $n = pq$ is derived, is not necessary. Since $n$ has been tested the parties can continue using $n$ to flip coins. Also note that the protocol listed above had been somewhat simplified and does not include message signatures. In practice however, due to legal reasons, signatures are important.\\
\\
\bfseries\large On Cheating\\
\\
\normalfont\normalsize The protocol procedure ensures that cheating is impossible. Firstly, it is not possible for Bob to deduce $\left(\frac{a}{n}\right)$ from $a^2 \bmod n$. The reason for this is that there are exactly four solutions to the equation $x^2 \equiv a^2 \bmod n$. Two of the solutions satisfy $\left(\frac{x}{n}\right) = 1$ and the other two satisfy $\left(\frac{x}{n}\right) = -1$. (Refer to Lemma 2.0.7 and Theorem 2.0.8).\\
\\
Secondly, it is not possible for Alice to substitute her original choice of $a$ with another solution to $x^2 \equiv a^2 \bmod n$ in order to manipulate the value of the Jacobi symbol. Alice is only able to find the other solutions to the congruence if she able factorize $n$. Alice cannont compute $x$ such that $x \nequiv a \bmod n$ and $x^2 \equiv a^2 \bmod n$ since the greatest common denominator of $\left(x + a\right)= p$ or $q$.

